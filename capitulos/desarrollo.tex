%%%%%%%%%%%%%%%%%%%%%%%%%%%%%%%%%%%%%%%%%%%%%%%%%%%%%%%%%%%%%%%%%%%%%%%%
% Plantilla TFG/TFM
% Escuela Politécnica Superior de la Universidad de Alicante
% Realizado por: Jose Manuel Requena Plens
% Contacto: info@jmrplens.com / Telegram:@jmrplens
%%%%%%%%%%%%%%%%%%%%%%%%%%%%%%%%%%%%%%%%%%%%%%%%%%%%%%%%%%%%%%%%%%%%%%%%

\chapter{Desarrollo}
\label{metodologia}


Debido a la imposibilidad temporal de realizar mediciones acústicas in situ en múltiples recintos se ha optado por realizar estas mediciones en dos recintos diferentes para después obtener unos modelos válidos en los programas de simulación acústica \textit{CATT-Acoustic} y \textit{EASE}.

En primer lugar se van a mostrar los dos recintos reales y los resultados obtenidos para después validar estos datos con los modelos configurados para la simulación.

Una vez validados los modelos se utilizarán estos para realizar modificaciones en los recintos, observar el comportamiento acústico y ajustar las ecuaciones de la teoría revisada corregida (sección \ref{teoriarevisadacorregida}) para igualar los campos acústicos calculados a los campos acústicos obtenidos mediante simulación (campo útil y campo perjudicial).
Con ello se obtendrá una relación de parámetros acústicos del recinto y los coeficientes de ajuste.


\section{Mediciones in situ}
\label{medicionesinsitu}
La mediciones se han realizado en dos recintos diferentes, ambos ubicados en la Universidad de Alicante (ambos recintos se describen ampliamente en los apartados siguientes):

\begin{description}
  \item[Aula OP/S003:] Aula ubicada en la Escuela de Óptica y Optometría.
  \item[Aula EP/0-26M:] Aula ubicada en la Escuela Politécnica Superior IV.
\end{description}




El equipo utilizado para la medición y posterior procesado está formado por una fuente omnidireccional dodecaédrica, micrófonos de medición de media pulgada y el software dBFA32 de 01dB utilizando señales MLS.

El software dBFA32 permite obtener la historia temporal (decaimiento del sonido respecto al tiempo) en fracciones de 1 milisegundo, estos datos serán utilizados para obtener los campos útil (0 a 50ms) y perjudicial (50 a $\infty$ ms) que serán comparados con los obtenidos en los modelos de CATT-Acoustic y EASE para validar las simulaciones.

Ambos recintos se han medido con mobiliario y sin él (mesas y sillas de estudiante) para comprobar la influencia de estos elementos en la distribución de los campos acústicos y, con dos ubicaciones de la fuente, en el centro y en una esquina.

Los datos se ofrecen con curvas de regresión, de tipo potencial para el campo útil y de tipo polinómica de grado uno para el campo perjudicial. La validación de los modelos vendrá dada por la comparación de las curvas de regresión de los campos.

\subsection{Aula OP/S003}

El aula OP/S003 está ubicada en la planta sótano de la Escuela de Óptica y Optometría de la Universidad de Alicante.

\begin{figure}[ht]
    \centering
    \begin{subfigure}[b]{0.4\textwidth}
    	\centering
        \includegraphics[width=0.9\linewidth]{archivos/optica2.jpg}
    \end{subfigure}
    ~ % Añadir el espacio deseado, si se deja la linea en blanco la siguiente subfigura ira en una nueva linea
    \begin{subfigure}[b]{0.4\textwidth}
    	\centering
        \includegraphics[width=0.9\linewidth]{archivos/optica1.jpg}
    \end{subfigure}
    \caption{Aula OP/S003.}\label{fotosoptica}
\end{figure}
\FloatBarrier 

Los detalles del recinto son:
\begin{itemize}
\itemsep0em
  \item \textbf{Dimensiones:} 18.9x9.6x2.8 $m$.
  \item \textbf{Volumen:} \textasciitilde500 $m^3$.
  \item \textbf{Superficie recinto:} \textasciitilde520 $m^2$.
  \item \textbf{Número de mesas:} 72 (dimensiones 1.20x0.45 $m$).
  \item \textbf{Número de sillas:} 144 (dimensiones 0.44x0.51 $m$).
\end{itemize}

Los materiales y sus absorciones según la literatura son:
\begin{itemize}
\itemsep0em
  \item Cerramientos laterales de hormigón armado: $\overline{\alpha}\approx0.02$.
  \item Cerramientos transversales de placa de yeso pintado: $\overline{\alpha}\approx0.03$.
  \item Suelo de terrazo: $\overline{\alpha}\approx0.02$.
  \item Techo registrable de escayola: $\overline{\alpha}\approx0.10$.
  \item Mobiliario de paneles multilaminares: $\overline{\alpha}\approx0.03$.
\end{itemize}

Las medidas de tiempo de reverberación del recinto por bandas de octava son:


