%%%%%%%%%%%%%%%%%%%%%%%%%%%%%%%%%%%%%%%%%%%%%%%%%%%%%%%%%%%%%%%%%%%%%%%%
% Plantilla TFG/TFM
% Escuela Politécnica Superior de la Universidad de Alicante
% Realizado por: Jose Manuel Requena Plens
% Contacto: info@jmrplens.com / Telegram:@jmrplens
%%%%%%%%%%%%%%%%%%%%%%%%%%%%%%%%%%%%%%%%%%%%%%%%%%%%%%%%%%%%%%%%%%%%%%%%

\chapter{Resultados}
\label{resultados}

A continuación se van a mostrar los coeficientes obtenidos mediante las simulaciones con EASE y los modelos validados (apartado \ref{modelosvalidados})  para ajustar la teoría revisada corregida (sección \ref{teoriarevisadacorregida}).

Para los casos sin modificaciones, tal como se encuentran los recintos reales, las curvas obtenidas con EASE y mediante la teoría revisada corregida se muestran en las figuras siguientes, para el resto de los casos sólo se van a mostrar los coeficientes obtenidos en los apartados posteriores.

En los casos con fuente en la esquina, tal como se ha visto en el apartado \ref{directividad}, el factor de directividad se debería ver modificado por los planos de las paredes aumentándolo a un factor $Q=4$, este hecho no se cumple en debido a que la fuente en esquina en ambos recintos se encuentra a más de 0.5 metros de las paredes en los recintos sin modificar y esta distancia se reduce o aumenta dependiendo del factor de escala.
Teniendo en cuenta que el coeficiente $C_D$ multiplica la ecuación de campo directo, al igual que lo hace el factor de directividad $Q$, si se mantiene en todos los cálculos $Q=1$, el coeficiente $C_D$ equivale al factor de directividad de la fuente.

\section{Recintos sin modificar}


\begin{table}[ht]
\centering
{\scalefont{0.8}
\begin{tabular}{@{}cccc@{}}
\toprule
Coeficiente & Simulación & Ecuación Eureqa & \% Error \\ \midrule
$\epsilon_L$ & 1.194 & 1.304 & 8.44 \\
$C_L$ & 0.803 & 0.871 & 7.81 \\
$\epsilon_E$ & -0.105 & - & - \\
$C_E$ & 2.717 & - & - \\ \bottomrule
\end{tabular}
}
\caption{Comparación de cálculos de coeficientes mediante ajuste de curvas y mediante ecuaciones obtenidas con el programa Eureqa para el aula EP/0-26M con factor de escala 2.6 y fuente en el centro.}
\label{tab:comparaecuscentro}
\vspace{-0.5cm}
\end{table}
\FloatBarrier






