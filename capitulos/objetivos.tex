%%%%%%%%%%%%%%%%%%%%%%%%%%%%%%%%%%%%%%%%%%%%%%%%%%%%%%%%%%%%%%%%%%%%%%%%
% Plantilla TFG/TFM
% Escuela Politécnica Superior de la Universidad de Alicante
% Realizado por: Jose Manuel Requena Plens
% Contacto: info@jmrplens.com / Telegram:@jmrplens
%%%%%%%%%%%%%%%%%%%%%%%%%%%%%%%%%%%%%%%%%%%%%%%%%%%%%%%%%%%%%%%%%%%%%%%%

\chapter{Objetivos}
\label{objetivos}

El objetivo principal es el estudio y desarrollo de alternativas al procesamiento de funciones hiperbólicas y trigonométricas de punto flotante según el estándar del IEEE 754. En concreto se centra el algoritmo CORDIC (COordinate Rotation DIgital Computer).

Se pretende encontrar ventajas e inconvenientes del algoritmo, soluciones propuestas y realizar una simulación e implementación de un algoritmo CORDIC básico en hardware utilizando numeración en punto flotante.

Como se verá en los siguientes apartados, la complejidad de algunas soluciones a los problemas inherentes del funcionamiento de CORDIC hace que el alcance del desarrollo del algoritmo sea simplemente una demostración teórica y práctica de como implementar un algoritmo de este tipo, dentro del marco académico.

Por último, hay que destacar que esta memoria(?) se centra principalmente en el aspecto del tratamiento de los datos



