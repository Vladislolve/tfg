%%%%%%%%%%%%%%%%%%%%%%%%%%%%%%%%%%%%%%%%%%%%%%%%%%%%%%%%%%%%%%%%%%%%%%%%
% Plantilla TFG/TFM
% Escuela Politécnica Superior de la Universidad de Alicante
% Realizado por: Jose Manuel Requena Plens
% Contacto: info@jmrplens.com / Telegram:@jmrplens
%%%%%%%%%%%%%%%%%%%%%%%%%%%%%%%%%%%%%%%%%%%%%%%%%%%%%%%%%%%%%%%%%%%%%%%%

\chapter{Objetivos}
\label{objetivos}

El objetivo principal es el estudio y desarrollo de alternativas al procesamiento de funciones hiperbólicas y trigonométricas de punto flotante según el estándar del IEEE 754. En concreto se centra el algoritmo CORDIC (Coordinate Rotation Digital Computer).

Se pretende encontrar ventajas e inconvenientes del algoritmo, soluciones propuestas y realizar una implementación, simulación e implementación(<----CAMBIAR ESTA PALABRA) en hardware.
2



