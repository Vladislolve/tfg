%%%%%%%%%%%%%%%%%%%%%%%%%%%%%%%%%%%%%%%%%%%%%%%%%%%%%%%%%%%%%%%%%%%%%%%
% Plantilla TFG/TFM
% Escuela Politécnica Superior de la Universidad de Alicante
% Realizado por: Jose Manuel Requena Plens
% Contacto: info@jmrplens.com / Telegram:@jmrplens
%%%%%%%%%%%%%%%%%%%%%%%%%%%%%%%%%%%%%%%%%%%%%%%%%%%%%%%%%%%%%%%%%%%%%%%%

\chapter{Introducción}


En 1959, Jack E. Volder publica el primer artículo del algoritmo que denominó \textit{The COordinate Rotation DIgital Computer}, o CORDIC. Este algoritmo fue originalmente como un algoritmo de propósito específico para dispositivos digitales que necesitan un funcionamiento en tiempo real. En concreto, este algoritmo fue diseñado para el cómputo del sistema de navegación del avión Convair B-56 Hustler para reemplazar el sistema análogo a uno digital para ser mas preciso y de tiempo real. 

El algoritmo publicado por Volder tenía como objetivo resolver funciones trigonométricas y la conversión de coordenadas rectangulares a polares, aunque desde el primer momento se estipula que este método puede ser usado para para el cómputo de otras funciones.

El aspecto más importante de CORDIC es la forma de resolver el problema. Solo requiere las operaciones de sumar, restar, movimiento de bits (\textit{bitshift}) y una \textit{lookup table} (LUT). Esto es importante, ya que trae muchas características que en un futuro dependen de esta simplicidad.


\section{Revisiones y mejoras de CORDIC}

J.S Walther, de la división de \textit{Hewlet-Packard Laboratories} (HP) publicó en 1971 un artículo con un algoritmo CORDIC que unificaba el cálculo de funciones elementales como la multiplicación, división, exponenciales, raíces cuadradas, entre otros, con las misma operaciones básicas especificadas anteriormente.

Además, en este artículo ya se hace mención de una unidad de hardware para el proceso de valores en punto flotante. El dispositivo diseñado en HP aceptaba valores de 48 bits y 32 bits en punto flotante.

El algoritmo de CORDIC fue usado por HP, Texas Instruments y otros para crear calculadoras digitales de un tamaño reducido. Un ejemplo de esto es la HP-35, la cual realizaba las funciones trigonométricas mediante este método.

\section{Presente y futuro de CORDIC}
asdf


\begin{itemize}
  \item Acústica arquitectónica (rama de este trabajo).
  \item Acústica ambiental.
  \item Electroacústica.
  \item Acústica submarina.
  \item Vibroacústica.
  \item Psicoacústica.
  \item Acústica fisiológica.
  \item Acústica fonética.
\end{itemize}

Existen muchas más ramas y algunas de ellas engloban otras ramas de conocimiento. La historia nos ha dejado grandes estudios y experimentos valiosos para forjar una gran base de conocimiento, en el punto siguiente se muestra una cronología resumida de algunos autores que dieron pasos para comprender mejor el apasionante mundo de la acústica.

\section{Cronología de la acústica}

La historia nos ha dado grandes nombres en el campo de acústica, los principales autores son\footnote{La mayoría de los datos de la cronología fueron recabados por Antonio Durá Domenech para su libro \textit{Temas de acústica} publicado por la Universidad de Alicante en 2005}:

\begin{description}
  \item[Galileo Galilei](1564-1642): Es considerado el precursor de la acústica moderna, sus aportaciones al campo de la acústica se basaron en el comportamiento y emisión de sonido de una cuerda dependiendo de sus parámetros físicos, las ondas estacionarias en cuerdas y la resonancia al excitarse una cuerda por efecto de otra.
  \item[Marin Mersenne](1588-1648): Aunque se considere a Galileo como el precursor de la acústica moderna, Marin fue un monje que realizó multitud de experimentos y estudios de gran calado por los que él es técnicamente el verdadero precursor de la acústica moderna. Determinó la frecuencia de distintas notas musicales y fue el primer autor en investigar la velocidad del sonido aunque no con mucho éxito.
  \end{description}
  

  
\section{Los campos acústicos}
  
Los campos acústicos definen de qué se compone el sonido en un punto concreto del espacio, por ejemplo, si nos encontramos en campo libre (no se produce ninguna reflexión del sonido) y emitimos un sonido, éste se propaga por el espacio como sonido directo o campo directo pero, si existen reflexiones ya no tenemos sólo el campo directo sino que encontramos sonido que ha sido reflejado desde otros puntos del espacio, a este sonido reflejado inicialmente se le denomina campo reverberante. 
 
 

























