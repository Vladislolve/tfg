%%%%%%%%%%%%%%%%%%%%%%%%%%%%%%%%%%%%%%%%%%%%%%%%%%%%%%%%%%%%%%%%%%%%%%%
% Plantilla TFG/TFM
% Escuela Politécnica Superior de la Universidad de Alicante
% Realizado por: Jose Manuel Requena Plens
% Contacto: info@jmrplens.com / Telegram:@jmrplens
%%%%%%%%%%%%%%%%%%%%%%%%%%%%%%%%%%%%%%%%%%%%%%%%%%%%%%%%%%%%%%%%%%%%%%%%

\chapter{Introducción}
\todo{Poner citas en todo el texto. Revisar}

En 1959, \citeauthor{volder_cordic_1959} publica el primer artículo del algoritmo que denominó \textit{The COordinate Rotation DIgital Computer}, o CORDIC. Este algoritmo fue originalmente como un algoritmo de propósito específico para dispositivos digitales que necesitan un funcionamiento en tiempo real. En concreto, este algoritmo fue diseñado para el cómputo del sistema de navegación del avión Convair B-56 Hustler para reemplazar el sistema analógico a uno digital para ser mas preciso y de tiempo real. 

El algoritmo publicado por Volder tenía como objetivo resolver funciones trigonométricas y la conversión de coordenadas rectangulares a polares, aunque desde el primer momento se estipula que este método puede ser usado para para el cómputo de otras operaciones matemáticas.

El aspecto más importante de CORDIC es la forma de resolver el problema. Solo requiere las operaciones de sumar, restar, movimiento de bits (\textit{bitshift}) y una \textit{lookup table} (LUT). Esto es importante, ya que trae muchas características que en un futuro dependen de esta simplicidad.


\section{Revisiones y mejoras de CORDIC}

\cite{walther_unified_1971}, de la división de \textit{Hewlet-Packard Laboratories} (HP) publicó en 1971 un artículo con un algoritmo CORDIC que unificaba el cálculo de funciones elementales como la multiplicación, división, exponenciales, raíces cuadradas, entre otros, con las misma operaciones básicas especificadas anteriormente.

Además, en este artículo ya se hace mención de una unidad de hardware para el proceso de valores en punto flotante. El dispositivo diseñado en HP aceptaba valores de 48 bits y 32 bits en punto flotante.

El algoritmo de CORDIC fue usado por HP, Texas Instruments y otros para crear calculadoras digitales de un tamaño reducido. Un ejemplo de esto es HP-35, la cual realizaba las funciones trigonométricas mediante este método. Además, CORDIC fue usado en algunos procesadores, ya sea como un \textit{coprocesador} o integrado dentro del mismo, como el Intel 8087 y algunos de sus posteriores generaciones, generalmente para reducir el número de puertas lógicas y complejidad de la FPU.


\section{Presente y futuro de CORDIC}
\todo{Poner mas cosas aquí. Citas}

Aunque el algoritmo no sea la mejor método para realizar las operaciones estipuladas anteriormente, sigue siendo muy atractivo por su simplicidad a la hora de implementarlo en hardware, ya que se puede usar el mismo algoritmo iterativo para todas las operaciones usando el diseño \textit{shift-add} básico. Muchos sistemas embebidos y FPGAs no tienen una unidad dedicada al procesamiento de punto flotante, por lo que lo hace un candidato ideal. \todo[inline]{2016 Study of CORDIC backing up development and reduced cost of FPGA and improved performance}

Los nuevos desarrollos de CORDIC se han centrado en mejorar el \textit{throughput}, la reducción de la complejidad del hardware necesitado para la implementación y la latencia propia del método.

Algunas de las aplicaciones(usos?) de CORDIC son:

\todo{La mayoría que ahora mismo son solo de 50 Years of CORDIC. Poner las actualizadas tmb.}

\begin{itemize}
  \item \textit{Direct Kinematics Solution} (DKS) o solución de cinemática directa para manipuladores de robots seriales. Las rotaciones y translaciones de los eslabones(links?) que generan nuevas coordenadas son calculadas por CORDIC. Un método similar es usado para la cinemática inversa.
  
  \item CORDIC tambien se ha usado como una unidad funcional de un procesador de robot para temas de redundancia y cálculo de detección de colisión.
  
  \item Dentro de los gráficos 3D nos encontramos con procesamiento intensivo geométrico de rotación de vectores 3D, luminosidad y interpolación, candidatos perfectos para uso del método.
  
  \item Descomposición QR. Hay variantes de CORDIC que ofrecen esta funcionalidad.
  
  \item  Dentro del procesamiento de señal podemos encontrar un gran número de implementaciones para el cálculo de \textit{Discrete Fourier transform} (DFT).\todo[inline]{Hay mas cosas que poner aquí de DFT}
  
  \item En el ámbito de la comunicación, CORDIC puede ser usado para la modulación digital y análoga. Dependiendo de los parámetros del método, es posible hacer cálculo digital de una multitud de modulaciones.
\end{itemize}

























