%%%%%%%%%%%%%%%%%%%%%%%%%%%%%%%%%%%%%%%%%%%%%%%%%%%%%%%%%%%%%%%%%%%%%%%%
% Plantilla TFG/TFM
% Escuela Politécnica Superior de la Universidad de Alicante
% Realizado por: Jose Manuel Requena Plens
% Contacto: info@jmrplens.com / Telegram:@jmrplens
%%%%%%%%%%%%%%%%%%%%%%%%%%%%%%%%%%%%%%%%%%%%%%%%%%%%%%%%%%%%%%%%%%%%%%%%


\chapter*{Resumen}
\thispagestyle{empty}
El método \gls{cordic} es un simple algoritmo que puede calcular de forma eficiente múltiples operaciones matemáticas, como funciones trigonométricas, raíces, multiplicaciones, divisiones, etc. solamente usando sumas, restas, movimiento de bits y una pequeña \gls{lut}.

El auge de procesamiento con punto flotante ha hecho que se tenga que revaluar los casos de uso del método e investigar y aportar variantes que usen este tipo de datos de forma eficiente. por lo que se ha realizado un estudio sobre las mejoras que se han ido realizando a lo largo de los 60 años y, además, se ha hecho hincapié en el posible uso de \gls{cordic} con punto flotante. 

Después de realizar un estudio extenso sobre los casos de uso e investigación actual sobre \gls{cordic}, se ha implementado tres variantes usando el lenguaje Verilog para evaluar las mejoras expuestas en los estudios expuestos anteriormente. El primero se basa en el propio método estándar, el segundo se añade \textit{pipelining} y un último con una conversión a punto flotante.

Tras la implementación, se ha visto una clara mejora de \textit{throughput} frente al \gls{cordic} básico, y se ha visto una mejora de precisión en \gls{cordic} en punto flotante, con el inconveniente de aumento del número de ciclos para procesar. También se ha visto un aumento de complejidad en la implementación de punto flotante, un punto negativo a la hora de tener en cuenta que el método tiene una filosofía de simplicidad.

La idea de investigar los casos de uso actuales del \gls{cordic} viene de Antonio Jimeno-Morenilla, que fue recomendado por uno de mis tutores, Marcelo Saval Calvo. Por recomendación de Antonio, Jose Luís Sánchez Romero se unió como segundo tutor para ayudar en el trabajo al tener un buen conocimiento de arquitecturas y haber trabajado con CORDIC anteriormente. Agradezco a todos los profesores que me han guiado en este trabajo y ofrecido cualquier apoyo que haya necesitado.



%\cleardoublepage %salta a nueva página impar
%\chapter*{Agradecimientos}

\thispagestyle{empty}
\vspace{1cm}


%\cleardoublepage %salta a nueva página impar
%% Aquí va la dedicatoria si la hubiese. Si no, comentar la(s) linea(s) siguientes
%\chapter*{}
%\label{dedicatoria}
%\setlength{\leftmargin}{0.5\textwidth}
%\setlength{\parsep}{0cm}
%\addtolength{\topsep}{0.5cm}
%\begin{flushright}
%\small\em{
%Una pequeña dedicatoria al líder supremo de Corea del Norte.
%%A toda persona que ha pasado por mi vida,
%%sin las cuales no sería la persona que soy hoy.
%}
%\end{flushright}


\cleardoublepage %salta a nueva página impar
% Aquí va la cita célebre si la hubiese. Si no, comentar la(s) linea(s) siguientes
\chapter*{}
\setlength{\leftmargin}{0.5\textwidth}
\setlength{\parsep}{0cm}
\addtolength{\topsep}{0.5cm}
\begin{flushright}
	\small\em{
		If something is hard for you to achieve \\
		do not suppose that it is beyond human capacity rather\\ 
		if something is possible and suitable for human beings \\
		consider that it is within your reach too.
		
	}
\end{flushright}
\begin{flushright}
	\small{
		Marcus Aurelius.
	}
\end{flushright}
\cleardoublepage %salta a nueva página impar