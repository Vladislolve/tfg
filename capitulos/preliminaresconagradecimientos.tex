%%%%%%%%%%%%%%%%%%%%%%%%%%%%%%%%%%%%%%%%%%%%%%%%%%%%%%%%%%%%%%%%%%%%%%%%
% Plantilla TFG/TFM
% Escuela Politécnica Superior de la Universidad de Alicante
% Realizado por: Jose Manuel Requena Plens
% Contacto: info@jmrplens.com / Telegram:@jmrplens
%%%%%%%%%%%%%%%%%%%%%%%%%%%%%%%%%%%%%%%%%%%%%%%%%%%%%%%%%%%%%%%%%%%%%%%%


\chapter*{Resumen}
\thispagestyle{empty}
El método \gls{cordic} es un simple algoritmo que puede calcular de forma eficiente múltiples operaciones matemáticas, como funciones trigonométricas, raíces, multiplicaciones, divisiones, etc. solamente usando sumas, restas, movimiento de bits y una \gls{lut}.

El auge de procesamiento con punto flotante ha hecho que se tenga que revaluar los casos de uso de \gls{cordic}, por lo que se ha realizado un estudio sobre las mejoras que se han ido realizando a lo largo de los 60 años y, además, se ha hecho hincapié en el posible uso de \gls{cordic} con punto flotante. 

Finalmente se ha realizado una implementación y comparativa de los distintos \gls{cordic} y se ha mostrado la dificultad de procesar con un estándar como es el IEEE 754, que estipula el punto flotante.



%\cleardoublepage %salta a nueva página impar
%\chapter*{Agradecimientos}

\thispagestyle{empty}
\vspace{1cm}


%\cleardoublepage %salta a nueva página impar
%% Aquí va la dedicatoria si la hubiese. Si no, comentar la(s) linea(s) siguientes
%\chapter*{}
%\label{dedicatoria}
%\setlength{\leftmargin}{0.5\textwidth}
%\setlength{\parsep}{0cm}
%\addtolength{\topsep}{0.5cm}
%\begin{flushright}
%\small\em{
%Una pequeña dedicatoria al líder supremo de Corea del Norte.
%%A toda persona que ha pasado por mi vida,
%%sin las cuales no sería la persona que soy hoy.
%}
%\end{flushright}


\cleardoublepage %salta a nueva página impar
% Aquí va la cita célebre si la hubiese. Si no, comentar la(s) linea(s) siguientes
\chapter*{}
\setlength{\leftmargin}{0.5\textwidth}
\setlength{\parsep}{0cm}
\addtolength{\topsep}{0.5cm}
\begin{flushright}
	\small\em{
		If something is hard for you to achieve \\
		do not suppose that it is beyond human capacity rather\\ 
		if something is possible and suitable for human beings \\
		consider that it is within your reach too.
		
	}
\end{flushright}
\begin{flushright}
	\small{
		Marcus Aurelius.
	}
\end{flushright}
\cleardoublepage %salta a nueva página impar