%%%%%%%%%%%%%%%%%%%%%%%%%%%%%%%%%%%%%%%%%%%%%%%%%%%%%%%%%%%%%%%%%%%%%%%%
% Plantilla TFG/TFM
% Escuela Politécnica Superior de la Universidad de Alicante
% Realizado por: Jose Manuel Requena Plens
% Contacto: info@jmrplens.com / Telegram:@jmrplens
%%%%%%%%%%%%%%%%%%%%%%%%%%%%%%%%%%%%%%%%%%%%%%%%%%%%%%%%%%%%%%%%%%%%%%%%

\chapter*{Abstract}
\thispagestyle{empty}
The separation in direct field and late field is correct although it has no use in the practice, due to the fact that the physiology of the human ear assembles the direct field and some of the first reflections making our brain believe that it all belongs to one only sound. This concept was developed by Hass (1951), where it is demonstrated that up to a certain instant of time all the sound received is integrated and the brain processes it as an only sound without reflexions and moreover it locates its origin at the place where the first sound has been received. The time defined by Hass for the spoken communication is 50 milliseconds, that is to say, all sound (direct and reflected) until 50 milliseconds have passed by is integrated by the auditive human system and understood as an only sound, this concept provoked that the nomenclature was used more conveniently for the direct field, early field (englobes all sound after the direct field until 50 ms) and late field (all sound after 50ms).

The late field defined as the field comprised from 50 milliseconds, arrives after the time in which the human ear integrates the sound as an only one and interferes with the sound that it was pretended to be heard. It is therefore that the acoustic fields can be named as useful field (direct field and early field) and useless field (late field from 50 milliseconds).

Taking into consideration the concepts of useful field and useless field and, understanding the importance of its relationship on the intelligibility of the word, we can comprehend the importance of the capacity to mathematically calculate the fields the most correct way. Nowadays there are a multitude of equations that, although being quite close to the experimental behaviour of the acoustic fields, are not completely correct and so they cannot be used to anticipate the behaviour. The last calculations developed by Barron and Lee (1988) are some of the few that considerate the temporary separation to mathematically calculate the useful and useless fields, although these calculations don’t correspond to the real behaviour as we will see later on this paper.

Given these difficulties we propose a modified calculation of the fields, based on the Barron and Lee equations and inspired by the search of and adjustment in the calculation of Sato and Bradley (2008), where, from a series of experimental measurements or by means of acoustic models, they are compared to the theoretical calculation of the useful and useless fields and adjustment coefficients are searched in order to finally bring out a valid theoretical calculation relating the obtained coefficients to the characteristics of the enclosure.



\chapter*{Resumen}
\thispagestyle{empty}
La división entre campo directo y campo reverberante es correcta aunque en la práctica carece de utilidad, debido a que la fisiología del oído humano agrupa el campo directo y algunas de las primeras reflexiones haciendo creer a nuestro cerebro que todo ello constituye un solo sonido. Este concepto fue desarrollado por \cite{Haas1949}, donde se demuestra que hasta cierto tiempo todo el sonido recibido se integra y el cerebro lo procesa como un único sonido sin reflexiones y además ubica el origen del sonido en el lugar de donde se ha recibido el primer sonido. El tiempo definido por \citeauthor{Haas1949} para la comunicación hablada es de 50 milisegundos, es decir, todo sonido (directo y reflejado) hasta transcurridos los 50 milisegundos es integrado por el sistema auditivo humano y entendido como un único sonido, este concepto produjo que se utilizara más convenientemente la nomenclatura de campo directo, campo temprano (engloba todo sonido después del campo directo hasta los 50 ms) y campo reverberante (todo sonido después de los 50 ms).
 
El campo reverberante definido como el campo comprendido desde los 50 milisegundos, llega después del tiempo en el que el oído humano integra el sonido como uno solo e interfiere en el sonido que se pretendía oír. Es por ello que los campos acústicos se pueden denominar como campo útil (campo directo y campo temprano) y campo perjudicial (campo reverberante desde los 50 ms).
 
Teniendo en cuenta el concepto de campo útil y campo perjudicial y, entendiendo la importancia de la relación de éstos en la inteligibilidad de la palabra, se puede comprender la importancia de poder calcular matemáticamente los campos de la forma más correcta posible. Actualmente existen multitud de ecuaciones que si bien se aproximan al comportamiento experimental de los campos acústicos no son del todo correctos y por tanto no se pueden utilizar para prever el comportamiento. Los últimos cálculos desarrollados por \cite{Barron1988} son de los pocos que tienen en cuenta una separación temporal para calcular matemáticamente el campo útil y perjudicial, aunque como se verá en este trabajo estos cálculos no se corresponden con el comportamiento real.

Ante esta problemática se propone un cálculo modificado de los campos, basado en las ecuaciones de \citeauthor{Barron1988} e inspirado en la búsqueda de un ajuste en el cálculo de \cite{Sato2008},  donde, a partir de una serie de medidas experimentales o mediante modelos acústicos, se comparan con el cálculo teórico de los campos útil y perjudicial y se buscan coeficientes de ajuste para obtener finalmente un cálculo teórico válido relacionando los coeficientes obtenidos con las características del recinto.



\cleardoublepage %salta a nueva página impar
\chapter*{Agradecimientos}

\thispagestyle{empty}
\vspace{1cm}

Este trabajo no habría sido posible sin el apoyo y confianza de Jenaro Vera Guarinos. Aún recuerdo aquel día en la terraza del bar de la Escuela Politécnica Superior cuando Jenaro se sentó sin previo aviso en la mesa donde estábamos unos amigos y yo y me dijo: ``Tengo algo que proponerte", no lo dudé y me tiré a la piscina, algo de lo que me alegraré toda la vida.
En la vida académica es difícil encontrar personas que se interesen y apoyen la carrera de otra persona sin buscar un mínimo provecho propio, pero en Jenaro he encontrado la excepción. La ayuda ha sido crucial, no el concepto de ayuda en el que una persona te libera de parte de la carga de trabajo, sino la de mostrar los caminos por donde dar las primeras pedaladas con ruedines y que después se aprende a prescindir de ellos. 
Confió en mí no sólo para este trabajo sino también, a fecha de hoy, para realizar varias comunicaciones para congresos sobre la misma temática de este trabajo.


También debo agradecer a mis padres la confianza que no debería merecer, no he sido la persona de la que se espera que llegue a la universidad y aun así han sido en todo momento un gran apoyo.

Han sido años de conocer a muchos compañeros de estudios, días y días de apoyo mutuo con tantos de ellos que no es posible mencionarlos a todos y todas, les agradezco la ayuda bidireccional aunque realmente no les menciono porque sino se les sube a la cabeza.


%Le dedico este trabajo a toda persona que ha pasado por mi vida, sin la cuales no sería la persona que soy hoy.

A todos y todas les dedico este trabajo.

%\cleardoublepage %salta a nueva página impar
%% Aquí va la dedicatoria si la hubiese. Si no, comentar la(s) linea(s) siguientes
%\chapter*{}
%\label{dedicatoria}
%\setlength{\leftmargin}{0.5\textwidth}
%\setlength{\parsep}{0cm}
%\addtolength{\topsep}{0.5cm}
%\begin{flushright}
%\small\em{
%Una pequeña dedicatoria al líder supremo de Corea del Norte.
%%A toda persona que ha pasado por mi vida,
%%sin las cuales no sería la persona que soy hoy.
%}
%\end{flushright}


\cleardoublepage %salta a nueva página impar
% Aquí va la cita célebre si la hubiese. Si no, comentar la(s) linea(s) siguientes
\chapter*{}
\setlength{\leftmargin}{0.5\textwidth}
\setlength{\parsep}{0cm}
\addtolength{\topsep}{0.5cm}
\cleardoublepage %salta a nueva página impar
