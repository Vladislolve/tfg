\chapter{Motivacion}
\label{motivacion}

CORDIC es un método desarrollado en los años 60 y ha sido utilizado en multiples ocasiones en proyectos de diferente gama, pero con la adición de la unidad de punto flotante (FPU) y sus posteriores mejoras en el procesamiento de las operaciones en punto flotante ademas de una reducción de coste de este, hizo bajar la popularidad de CORDIC ya que el procesamiento de CORDIC en software es mas lento y por lo tanto, no tan atractivo a la hora de elegir el método.

Aun así, CORDIC puede ocupar un espacio el cual una FPU no lo podría. Múltiples métodos y algoritmos se han quedado atras en el mundo académico (y en el mundo real/industrias??), pero CORDIC sigue siendo estudio y despues de mas de 50 años, se siguen publicando artículos relacionados con el método, por lo que hay una razón por la que sería interesante estudiar sus posibles aplicaciones.

El método engloba una gran cantidad de diferentes algoritmos que se construyen de la misma manera y emplean las mismas operaciones estipuladas por Jack. E Volder para realizar diferentes operaciones. Esta flexibilidad permite que el algoritmo evolucione, haya mejoras y no se quede atras en la revolución informática(???)

Para finalizar, CORDIC permite un aprendizaje de los lenguajes de descripción de hardware y la implementación del código en hardware especializado, como por ejemplo una FPGA, ya que es un método fácil de entender e implementar en hardware en su definición mas básica (cálculo de funciones sin() y cos()).
