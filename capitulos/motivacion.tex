\chapter{Motivación}
\label{motivacion}

CORDIC es un método desarrollado en los años 60 que ha sido utilizado en múltiples ocasiones en proyectos de diferente gama, pero con la adición de la unidad de punto flotante (FPU) y sus posteriores mejoras en el procesamiento de las operaciones en punto flotante ademas de una reducción de coste de este, hizo bajar la popularidad de CORDIC ya que el procesamiento de CORDIC posee una gran latencia. Además, en software es mas lento y por lo tanto, no tan atractivo a la hora de elegir el método.

Aun así, CORDIC puede ocupar un espacio el cual una FPU no es lo mas óptimo. Múltiples métodos y algoritmos se han quedado atrás en el mundo académico y no han sido usados para solucionar problemas reales, pero CORDIC sigue siendo estudiado y después de mas de 50 años se siguen publicando artículos relacionados con el método, por lo que hay una razón por la que sería interesante estudiar sus posibles aplicaciones.

El método engloba una gran cantidad de diferentes algoritmos que se construyen de la misma manera y emplean las mismas operaciones estipuladas por Jack. E Volder. Esta flexibilidad permite el uso del algoritmo en diferentes áreas de la informática, como el procesamiento de imágenes, comunicación o robótica, entre otros.

El ecosistema de FPGAs está ahora mismo en alza gracias a la reducción de coste de los componentes y la mejora de las especificaciones. Además, se ha visto un gran número de herramientas de código abierto que anteriormente no existía, y se basaba mayoritariamente en herramientas de desarrollo software multiplataforma y dispositivos hardware propietarios. Un claro ejemplo son las \textit{crowdfunded} FPGAs que hay en el mercado, que ofrecen unas características muy interesantes a un precio relativamente bajo.\todo[inline]{Referencias a algunas de estas FPGAs.} Esto permite poner vista a nuevos proyectos con FPGAs que anteriormente eran muy costosos ya que se ve una clara bajada de coste de cómputo en un futuro muy cercano.

Para finalizar, CORDIC permite un aprendizaje de los lenguajes de descripción de hardware y la implementación del código en hardware especializado, como por ejemplo una FPGA, ya que es un método fácil de entender e implementar en hardware en su definición mas básica (cálculo de funciones sin() y cos()).
